\documentclass[lang=cn,newtx,10pt,scheme=chinese]{elegantbook}

\title{数学之桥:从初等数学到微积分}
\subtitle{习题答案}

\author{Siliconet}
\date{始于2023年}
\version{1.0}
\bioinfo{模板}{ElegantBook}

\extrainfo{你站在桥上看风景,看风景人在楼上看你。明月装饰了你的窗子,你装饰了别人的梦。}

\setcounter{tocdepth}{3}

\cover{cover.jpg}

% 本文档命令
\usepackage{array}
\newcommand{\ccr}[1]{\makecell{{\color{#1}\rule{1cm}{1cm}}}}

% 修改标题页的橙色带
\definecolor{customcolor}{RGB}{255,255,255}
\colorlet{coverlinecolor}{customcolor}
\usepackage{cprotect}

\addbibresource[location=local]{reference.bib} % 参考文献,不要删除

\begin{document}

\maketitle
\frontmatter

\tableofcontents

\mainmatter

\chapter{集合论初步}
\begin{exercise}
  $\in\enspace\notin\enspace\notin\enspace\in\enspace\notin$.
\end{exercise}
\begin{exercise}
  无限集;有限集;无限集。
\end{exercise}
\begin{exercise}
  $\{2,-2\}$;$\{-2,-1,0,1,2\}$;$\{1\}$.
\end{exercise}
\begin{exercise}
  (1)$\{(x,y)\mid x>0,y>0\}$;(2)$\{\text{矩形}\}$;(3)$\{x\mid x=5^n,0\leq n\leq 4,n\in\mathbb{N}\}$.
\end{exercise}
\begin{exercise}
  $\{(2,1,3)\}$.
\end{exercise}
\begin{exercise}
  $-1$或$-8$。
\end{exercise}
\begin{exercise}
  (1)线段$AB$的垂直平分线;(2)以$O$为圆心,$3cm$为半径的圆;(3)以$O$为圆心,内半径为2,外半径为3的圆环。
\end{exercise}
\begin{exercise}
  (1)$\{y|y\geqslant0\}$或$[0,+\infty)$;(2)$\{y|y\neq0\}$;(3)$\{y\text{轴上的点}\}$
\end{exercise}
\begin{exercise}
  $m\geqslant2\text{或}m=1$.
\end{exercise}
<<<<<<< HEAD
\begin{exercise}
  略。
\end{exercise}
\begin{exercise}
  略。
\end{exercise}
\begin{exercise}
  不相等.$\varnothing\subseteq\{0\}.$
\end{exercise}
\begin{exercise}
  略。
\end{exercise}
\begin{exercise}
  $\overline{A}=\{4,5\};\overline{A}\cap B=\{3\};A\cup\overline{B}=\{1,2,3,4,6\}.$
\end{exercise}
\begin{exercise}
  $m$的值为$\sqrt{3}$或$-\sqrt{3}$或$0$.
\end{exercise}
\begin{exercise}
  $\{4\};\{1,2,4,5,6,7,8,9,10\};\{3\};\{1,2,3,5,6,7,8,9,10\};\varnothing ;\{1,2,3,4,5,6,7,8,9,10\}.$
\end{exercise}
\begin{exercise}
  $[4,+\infty).$
\end{exercise}
\begin{exercise}
  略。
\end{exercise}
\begin{exercise}
  略。
\end{exercise}
\begin{exercise}
  $\{-3,-1,1,5\}$.
\end{exercise}
\begin{exercise}
  (1)设$s=m_1^2+n_1^2,t=m_2^2+n_2^2$,其中$m_1,m_2,n_1,n_2$均为整数。由此可得
  $$\begin{aligned}
    st& =(m_1^2+n_1^2)(m_2^2+n_2^2)  \\
    &=m_1^2m_2^2+n_1^2n_2^2+m_1^2n_2^2+m_2^2n_1^2 \\
    &=(m_1m_2+n_1n_2)^2+(m_1n_2-m_2n_1)^2
    \end{aligned}$$
  因此$st\in A$,得证。
  
  (2)由(1)知$st \in A$,设$st=m^2+n^2$,其中$m,n$都是整数。由于$t\neq 0$,可得$$\frac st=\frac{st}{t^2}=\frac{m^2+n^2}{t^2}=\left(\frac mt\right)^2+\left(\frac nt\right)^2$$
  得证。
\end{exercise}


=======
>>>>>>> 0c7b118cbaf4540e3322adc4c628298ae0bc0732
\end{document}
