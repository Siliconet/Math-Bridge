\documentclass[lang=cn,newtx,10pt,scheme=chinese]{elegantbook}

\title{数学之桥:从初等数学到微积分}
\subtitle{习题答案}

\author{Siliconet}
\date{始于2023年}
\version{1.0}
\bioinfo{模板}{ElegantBook}

\extrainfo{你站在桥上看风景,看风景人在楼上看你.明月装饰了你的窗子,你装饰了别人的梦.}

\setcounter{tocdepth}{3}

\cover{cover.jpg}

% 本文档命令
\usepackage{array}
\newcommand{\ccr}[1]{\makecell{{\color{#1}\rule{1cm}{1cm}}}}

% 修改标题页的橙色带
\definecolor{customcolor}{RGB}{255,255,255}
\colorlet{coverlinecolor}{customcolor}
\usepackage{cprotect}

\addbibresource[location=local]{reference.bib} % 参考文献,不要删除

\begin{document}

\maketitle
\frontmatter

\tableofcontents

\mainmatter

\chapter{集合论初步}
\begin{exercise}
  $\in\enspace\notin\enspace\notin\enspace\in\enspace\notin$.
\end{exercise}
\begin{exercise}
  无限集;有限集;无限集.
\end{exercise}
\begin{exercise}
  $\{2,-2\}$;$\{-2,-1,0,1,2\}$;$\{1\}$.
\end{exercise}
\begin{exercise}
  (1)$\{(x,y)\mid x>0,y>0\}$;(2)$\{\text{矩形}\}$;(3)$\{x\mid x=5^n,0\leq n\leq 4,n\in\mathbb{N}\}$.
\end{exercise}
\begin{exercise}
  $\{(2,1,3)\}$.
\end{exercise}
\begin{exercise}
  $-1$或$-8$.
\end{exercise}
\begin{exercise}
  (1)线段$AB$的垂直平分线;(2)以$O$为圆心,$3cm$为半径的圆;(3)以$O$为圆心,内半径为2,外半径为3的圆环.
\end{exercise}
\begin{exercise}
  (1)$\{y|y\geqslant0\}$或$[0,+\infty)$;(2)$\{y|y\neq0\}$;(3)$\{y\text{轴上的点}\}$
\end{exercise}
\begin{exercise}
  $m\geqslant2\text{或}m=1$.
\end{exercise}
\begin{exercise}
  983.
\end{exercise}
\begin{exercise}
  略.
\end{exercise}
\begin{exercise}
  略.
\end{exercise}
\begin{exercise}
  不相等.$\varnothing\subseteq\{0\}.$
\end{exercise}
\begin{exercise}
  略.
\end{exercise}
\begin{exercise}
  $\overline{A}=\{4,5\};\overline{A}\cap B=\{3\};A\cup\overline{B}=\{1,2,3,4,6\}.$
\end{exercise}
\begin{exercise}
  $\{4\};\{1,2,4,5,6,7,8,9,10\};\{3\};\{1,2,3,5,6,7,8,9,10\};\varnothing ;\{1,2,3,4,5,6,7,8,9,10\}.$
\end{exercise}
\begin{exercise}
  $[4,+\infty).$
\end{exercise}
\begin{exercise}
  略.
\end{exercise}
\begin{exercise}
  (1)(2)略.
  (3)60.
\end{exercise}
\begin{exercise}
  $\{-3,-1,1,5\}$.
\end{exercise}
\begin{exercise}
  (1)设$s=m_1^2+n_1^2,t=m_2^2+n_2^2$,其中$m_1,m_2,n_1,n_2$均为整数.由此可得
  $$\begin{aligned}
    st& =(m_1^2+n_1^2)(m_2^2+n_2^2)  \\
    &=m_1^2m_2^2+n_1^2n_2^2+m_1^2n_2^2+m_2^2n_1^2 \\
    &=(m_1m_2+n_1n_2)^2+(m_1n_2-m_2n_1)^2
    \end{aligned}$$
  因此$st\in A$,得证.
  
  (2)由(1)知$st \in A$,设$st=m^2+n^2$,其中$m,n$都是整数.由于$t\neq 0$,可得$$\frac st=\frac{st}{t^2}=\frac{m^2+n^2}{t^2}=\left(\frac mt\right)^2+\left(\frac nt\right)^2$$
  得证.
\end{exercise}
\begin{exercise}\label{exer:202405021028}
  试着举出一个真命题,它的逆命题不是真命题.
\end{exercise}
\begin{solution}
  略.
\end{solution}
\begin{exercise}
  在命题“三角形的内角和等于$180^{\circ}$”中,条件和结论分别是什么?
\end{exercise}
\begin{solution}
  条件:一个平面图形是三角形;结论:这个平面图形的内角和为$180^{\circ}$.
\end{solution}
\begin{example}\label{exp:202406231441}
  幼儿园的小朋友都知道,$4>2$,所以,$x>4\Rightarrow x>2$.
\end{example}
\begin{problem}\label{202406262000}
  参考例\ref{exp:202406231441},思考怎样从集合的观点理解推出关系.
\end{problem}
\begin{solution}
  略.
\end{solution}
\begin{exercise}\label{202406262010}
  判断下列各题中,$p$是否是$q$的充分条件,$q$是否是$p$的必要条件:

  (1)$p:x\in\mathbb{Z},q:x\in\mathbb{R}$;

  (2)$p:a=b,q:ac=bc$;

  (3)$p:$两个三角形全等,$q:$两个三角形面积相等.
\end{exercise}
\begin{solution}
  (1)是;
  (2)是;
  (3)是.
\end{solution}
\begin{exercise}\label{202406262101}
  判断下列“若p,则q”形式的命题当中,哪些命题中的p是q的充分条件,或者说q是p的必要条件?

  (1)若四边形的两组对角分别相等,则这个四边形是平行四边形;

  (2)若四边形为菱形,则这个四边形的对角线互相垂直;

  (3)若$x,y$为无理数,则$xy$为无理数;

  (4)若直线$l$与$\odot O$有且仅有一个交点,则$l$为$\odot O$的一条切线;

  (5)若$x,y$都能被5整除,则$x+y$也能被5整除;

  (6)如果$\dfrac{a}{b}=\dfrac{c}{d}$,则$\dfrac{a+b}{b}=\dfrac{c+d}{d}$.
\end{exercise}
\begin{solution}
  (1)(2)(4)(5)(6)
\end{solution}
\begin{exercise}\label{RJB_P26}
  判断下列命题的真假:

  (1)存在两个无理数,它们的乘积是有理数;

  (2)没有一个无理数不是实数;

  (3)集合$A$是集合$A\cup B$的子集;

  (4)集合$A\cap B$是集合$A$的子集;

  (5)如果$\dfrac ab=\dfrac cd$,则$\dfrac{a+b}b=\dfrac{c+d}{d}$;

  (6)一元三次方程都有三个不同的实数根;
  
  (7)如果$\dfrac ab=\dfrac cd\neq 1$,则$\dfrac a{b-a}=\dfrac c{d-c}$.
\end{exercise}
\begin{solution}
  真;真;真;真;真;假;真.
\end{solution}
\begin{exercise}\label{202407081520}
  证明:$\sqrt{3}$是无理数.
\end{exercise}
\begin{solution}
  略.
\end{solution}
\begin{exercise}\label{BJ4Z_Algebra1_P28.2}
  用“充要”“充分不必要”“必要不充分”“既不充分也不必要”填空(本题中默认$a\in\mathbb{R},b\in\mathbb{R}$):

  (1)$a^2=b^2$是$a=b$的\underline{\hbox to 15mm{}}条件;

  (2)$a^3=b^3$是$a=b$的\underline{\hbox to 15mm{}}条件;

  (3)使$ab=0$的充分条件是\underline{\hbox to 15mm{}};

  (4)使$ab=0$的必要条件是\underline{\hbox to 15mm{}};

  (5)使$ab\neq 0$的充要条件是\underline{\hbox to 15mm{}};

  (6)“$a>2$”是“$a\geqslant2$”的\underline{\hbox to 15mm{}}条件;

  (7)“$a、b$不全是0”是“$a、b$全不是0”的\underline{\hbox to 15mm{}}条件;

  (8)“$\vert a-2\vert\neq2-a$”是“$a\geqslant2$”的\underline{\hbox to 15mm{}}条件.
\end{exercise}
\begin{solution}
  必要不充分;充要;$a=0$(仅举一例);$a=0$或$b=0$;$a\neq 0$且$b\neq 0$;充分不必要;既不充分也不必要;充要.
\end{solution}

\begin{exercise}\label{2017_XJ_bx1_P23.8}
  已知$p$是$r$的充分条件,$r$是$q$的必要条件,同时也是$s$的充分条件,$q$是$s$的必要条件,那么:
\end{exercise}

\begin{enumerate}
  \item $s$是$p$的什么条件?
  \item $p$是$q$的什么条件?
  \item 在$p,q,r,s$中,哪几对互为充要条件?
\end{enumerate}

\begin{solution}
  必要条件;充分条件;$r$与$q$,$r$与$s$,$q$与$s$.
\end{solution}

\begin{exercise}\label{2017_RJB_bx1_P36.B5}
  已知$A=(-\infty,a]$,$B=(-\infty,3)$,且$x\in A$是$x\in B$的充分不必要条件,求$a$的取值范围.
\end{exercise}

\begin{solution}
  $(-\infty,3)$.
\end{solution}

\begin{exercise}\label{zhw2000_g1_P51.78}
  求证:关于$x$的方程$ax^2+bx+1=0$有一根为1的充要条件是$a+b+c=0$.
\end{exercise}

\begin{solution}
  必要性证明略,下面证明充分性.

  若$a+b+c=0$,则原方程可化为$ax^2+bx-(a+b)=0$,即$(x-1)(ax+a+b)=0$,由此可知原方程有一根为1.充分性证毕.
\end{solution}

\hspace*{\fill}
\begin{itemize}
  \item 两个命题互为逆否命题,它们的真假性相同;
  \item 两个命题为互逆命题或互否命题,它们的真假性无关.
\end{itemize}
\begin{problem}
  举出这样的例子.
\end{problem}
\begin{solution}
  略.
\end{solution}

\begin{exercise}
  用数学符号写出下列命题并判断真假:
\end{exercise}

\begin{enumerate}
  \item 所有实数的平方都是正数;
  \item 存在两个无理数,它们的乘积是有理数;
  \item 三个连续整数的乘积是6的倍数;
  \item 大于3的自然数是不等式$x^2>10$的解.
\end{enumerate}

\begin{solution}
  $\forall x\in\mathbb{R},x^2>0$,假;$\exists a,b\in\complement_{\mathbb{R}}\mathbb{Q},ab\in\mathbb{Q}$,真;$\forall a\in\mathbb{N},a(a-1)(a+1)=6k\text{且}k\in\mathbb{Z}$,假;$\forall x>3\text{且}x\in\mathbb{N},x^2>10$,真.
\end{solution}

\begin{exercise}
  判断下列命题的真假,并写出这些命题的否定(请注意这里有些命题\textcolor{blue}{省略了全称量词}):
\end{exercise}

\begin{enumerate}
  \item 至少有一个整数$n$,$n^2+1$是4的倍数;
  \item 每个二次函数的图像都是轴对称图形;
  \item 存在一个四边形,它的四个顶点不共圆;
  \item 若$x>1$,则$2x+1>5$;
  \item 若四边形为等腰梯形,则这个四边形的对角线相等;
  \item $\forall x{\in}\mathbb{R}$,$\dfrac1{x^2+1}<1$;
  \item $\forall a,b{\in}\mathbb{R}$,$a^3+b^3=(a+b)(a^2-ab+b^2)$.
\end{enumerate}

\begin{solution}
  
\end{solution}

\begin{enumerate}
  \item 假;对所有的整数$n$,$n^2+1$不是4的倍数;
  \item 真;至少有一个二次函数的图像不是轴对称图形;
  \item 真;任意一个四边形的四个顶点共圆;
  \item 假;存在$x>1$,使得$2x+1\leqslant 5$;
  \item 真;存在一个为等腰梯形的四边形,它的对角线不相等;
  \item 假;$\exists x\in\mathbb{R},\dfrac1{x^2+1}\geqslant 1$;
  \item 真;$\exists a,b{\in}\mathbb{R}$,$a^3+b^3\neq(a+b)(a^2-ab+b^2)$.
\end{enumerate}

\begin{exercise}
  判断下列命题的真假:
\end{exercise}

\begin{enumerate}
  \item $\forall x\in(-1, 2), x\in[-1, 2)$;
  \item $\forall x\in\mathbb{Q}, |x|+x\geqslant0$;
  \item $\exists x,y\in\mathbb{Z},3x-2y=10$;
  \item 集合$A$是集合$A\cup B$的子集;
  \item 集合$A\cap B$是集合$A$的子集;
  \item 存在有序整数组$(x,y)$满足$xy=x+y$.
\end{enumerate}

\begin{solution}
  真;真;真;真;真;真.
\end{solution}

\begin{exercise}
  用联结词“且”和“或”分别联结下面各组命题组成新命题,并判断它们的真假:
\end{exercise}

\begin{enumerate}
  \item $p$:$17<20$;$q$:$17=20$.
  \item $p$:平行四边形对角线互相平分;$q$:平行四边形的对角线相等.
  \item $p$:4是素数;$q$:4不是奇数;
  \item $p$:能被5整除的整数的个位数一定为5;$q$:能被5整除的整数的个位数一定为0.
\end{enumerate}

\begin{solution}
  略;略;略;能被5整除的整数的个位数一定为5或一定为0,假.
\end{solution}

\begin{exercise}
  写出下列命题的非:
\end{exercise}

\begin{enumerate}
  \item $\forall x\in\mathbb{R},3x=2x+x$;
  \item $\exists x\in\mathbb{N},x^2=x+2$;
  \item 至少有一个锐角$\alpha$,使$\sin\alpha=0$;
  \item $a,b$都是有理数.
\end{enumerate}\

\begin{solution}
  略;略;对于任意的锐角$\alpha$,$\sin\alpha\neq 0$;$a,b$不全是有理数.
\end{solution}

\begin{exercise}
  写出下列命题的逆命题、否命题和逆否命题并判断真假:
\end{exercise}

\begin{enumerate}
  \item 线段的垂直平分线上的点到这条线段两个端点的距离相等;
  \item 矩形的对角线相等;
  \item 如果$x=y=0$,那么$(x-y)(x+y)=0$;
  \item 对顶角相等.
\end{enumerate}

\begin{solution}
  略.
\end{solution}

\begin{exercise}\label{2003RJA_xx2-1_P7.exp4}
  证明:若$x^2+y^2=0$,则$x=y=0$.
\end{exercise}

\begin{solution}
  取逆否命题即可.
\end{solution}

\chapter{函数}

\begin{exercise}
  定义2.1中的$B$是映射$f$的值域吗?
\end{exercise}

\begin{solution}
  不一定是.参见前面的例子.
\end{solution}

\begin{exercise}\label{HS2FZ_lkb1_P34_exp.4,BJSZ_Algebra1_P40}
  下列对应关系是不是从$A$到$B$的映射?
\end{exercise}

\begin{enumerate}
  \item $A=\mathbb{R},B=\mathbb{R}^{+},f:x\rightarrow\left|x\right|$;
  \item $A=\mathbb{R},B=\left\{y\mid y\geqslant 0\right\},f:x\rightarrow y=x^2$;
  \item $A=\{x\mid x\in\mathbb{R},x\geqslant0\},B=\mathbb{R}$,对应法则$f$:任意$X\in A$都对应于其平方根.
\end{enumerate}

\begin{solution}
  不是(取$x=0$);是;不是.
\end{solution}

\begin{exercise}
  设集合$A={a,b},B={m,n}$,从$A$到$B$可以建立多少种不同的映射?其中有多少种是一一映射?
\end{exercise}

\begin{solution}
  4;2.
\end{solution}

\begin{exercise}
  在下列给定的集合$A$和$B$间建立一一映射.
\end{exercise}

\begin{enumerate}
  \item $A=\left\{x\mid 0\leqslant x\leqslant 2\right\},B=\left\{y\mid 0\leqslant x\leqslant 4\right\}$;
  \item $A=[a,b],B=[0,1]$;
  \item $A=(0,1),B=(1,+\infty)$.
\end{enumerate}

\begin{solution}
  略.
\end{solution}

\begin{exercise}
  求下列函数的自然定义域:
\end{exercise}

\begin{spacing}{1.8}
  \begin{enumerate}
    \item $f(x)=\sqrt{x-1}\cdot\sqrt{x+1}$;
    \item $f(x)=\dfrac{\sqrt[3]{4x+8}}{\sqrt{3x-2}}$;
    \item $f(x)=\dfrac1{\sqrt[3]{3x+6}}+\dfrac1x$.
  \end{enumerate}


  \begin{solution}
    $[1,+\infty);(\dfrac23,+\infty);(-\infty,-2)\cup(-2,0)\cup(0,+\infty)$.
  \end{solution}

  \begin{exercise}
    求下列函数的值域:
  \end{exercise}

  \begin{enumerate}
    \item $f(x)=x^2+4x-5$;
    \item $f(x)=5-\sqrt{-x^{2}+2x+3}$
    \item $f(x)=x+\sqrt{1-2x}$
    \item $f(x)=\dfrac{5x^{2}+9x+4}{x^{2}-1}$
    \item $f(x)=\mid x-1\mid+\mid x+4\mid $
  \end{enumerate}
\end{spacing}

\begin{solution}
  $[-9,+\infty);[3,5];(-\infty,1];(-\infty,\dfrac12)\cup(\dfrac12,5)\cup(5,+\infty);[5,+\infty)$.
\end{solution}

\begin{exercise}
  已知函数$f(x)=3x^2+4x-8$,求$f(2)$,$f(a)$,$f(a-2)$,$f(a)-f(2)$的值.
\end{exercise}

\begin{solution}
  12;$3a^2+4a-8$;$3a^2-8a-4$;$3a^2+4a-20$.
\end{solution}

\begin{exercise}\label{2017RJA.P74.17}
  是否存在函数$f(x),g(x)$满足条件:
\end{exercise}

\begin{enumerate}
  \item 定义域相同,值域相同,但是对应关系不同;
  \item 值域相同,对应关系相同,但是定义域不同.
\end{enumerate}

\begin{solution}
  存在;存在.
\end{solution}

\begin{exercise}
  将$y=f(x)$(这是原函数,这里的$x$和下面的$x$并不一样)的图像向左平移$h$个单位长度,则有
$\begin{cases}
  x=x'-h
  \\y=y'
\end{cases}$,也就是
$\begin{cases}
  x'=x+h
  \\y'=y
\end{cases}$,
由于$y'=f(x')$,把$x',y'$代回去,可知变换后的函数图像为$y=f(x+h)$的图像.

同理可得:

\begin{enumerate}
  \item 将$y=f(x)$的图像向右平移$h$个单位长度得到$y=f(x-h)$的图像;
  \item 将$y=f(x)$的图像向下平移$h$个单位长度得到$y=f(x)-h$的图像;
  \item 将$y=f(x)$的图像向上平移$h$个单位长度得到$y=f(x)+h$的图像.
\end{enumerate}

  证明上述规律.
\end{exercise}

\begin{solution}
  略.
\end{solution}

\begin{exercise}\label{2017RJB.P94.8.changed}
  已知函数$f(x+2)=3x-7$,求$f(1),f(x)$.
\end{exercise}

\begin{solution}
  -10;$f(x)=3x-13$.
\end{solution}

\begin{exercise}
  设函数$f(x)=x^2-7,g(x)=4x+8$,求$f(g(x)),g(f(x))$.
\end{exercise}

\begin{solution}
  略.
\end{solution}

\begin{exercise}
  作出下列函数的大致图像:
\end{exercise}

\begin{spacing}{1.8}
  \begin{enumerate}
    \item $f(x)=\dfrac12 x^2-x+2$;
    \item $f(x)=x^3-2x+7$;
    \item $f(x)=\dfrac{1}{10}x^4-x^2-5$;
    \item $f(x)=\dfrac{x-1}{2x+1}$;
    \item $f(x)=\sqrt{x-1}$;
    \item $f(x)=\left|x\right|^2+\left|x\right|+2$;
    \item $f(x)=\dfrac{1-\left|x\right|}{\left|1-x\right|}$;
    \item $f(x)=\dfrac{1}{1+x^2}$(箕舌线);
    \item $f(x)=\dfrac{2x}{1+x^2}$(牛顿蛇形线);
    \item $f(x)=1+\sqrt{x^{2}+2x}$.
  \end{enumerate}
\end{spacing}

\begin{solution}
  略.(你可以用工具软件验证一下)
\end{solution}

\begin{exercise}\label{BJSZ.Algebra1.P58-59.changed}
  下列各题中,函数$f_1(x)$的图像经过怎样的变换可得到$f_2(x)$的图像?
\end{exercise}

\begin{spacing}{1.7}
  \begin{enumerate}
    \item $f_{1}(x)=2(x+2)^{2}, f_{2}(x)=2(x-2)^{2}+1$;
    \item $f_{1}(x)=\dfrac{1}{x}, f_{2}(x)=\dfrac{1}{2x+3}$;
    \item $f_{1}\left(x\right)=\left(x-1\right)^{2}-3,\quad f_{2}\left(x\right)=\left|\left(x-1\right)^{2}-3\right|$;
    \item $f_{1}(x)=(x-1)^{2}-3,f_{2}(x)=(|x|-1)^{2}-3$.
  \end{enumerate}
\end{spacing}

\begin{solution}
  略.
\end{solution}

\begin{exercise}
  定义域为$\mathbb{R}$的函数$f(x)$满足以下条件,分别写出对应条件下$f(x)$的图像的一条对称轴:
\end{exercise}

\begin{enumerate}
  \item $f(1-x)=f(1+x)$;
  \item $f(x-1)=f(2-x)$;
  \item $f(b+x)=f(c-x)$.
\end{enumerate}

\begin{solution}
  $x=1;x=\dfrac12;x=\dfrac{b+c}{2}.$
\end{solution}

\begin{exercise}
  定义域为$\mathbb{R}$的函数$f(x)$满足以下条件,分别写出对应条件下$f(x)$关于哪一点对称:
\end{exercise}

\begin{enumerate}
  \item $f(2-x)=-f(2+x)$;
  \item $f(2-x)=-f(1+x)$;
  \item $f(a-x)+f(c+x)=0$;
  \item $f(a-x)+f(c+x)=2b$.
\end{enumerate}

\begin{spacing}{1.8}
  \begin{solution}
    (2,0);($\dfrac32$,0);($\dfrac{a+c}{2}$,0);($\dfrac{a+c}{2}$,b).
  \end{solution}
\end{spacing}

\begin{exercise}
  已知函数$f(x)$,若存在$x\in\mathbb{R}$使得$f(x)=x$,则称$x$为$f(x)$的不动点.
\end{exercise}

\begin{enumerate}
  \item 求证:$f(x)$的不动点也是$f(f(x))$的不动点;
  \item 求证:若$f(f(x))$有唯一的不动点,则$f(x)$也有唯一的不动点.
\end{enumerate}

\begin{solution}
  略.
\end{solution}

\begin{exercise}\label{ASJC_G1_P22.5}
  求证:对任意的实数$p$,抛物线$y=2x^2-px+4p+1$恒过一定点.
\end{exercise}

\begin{solution}
  略.
\end{solution}

\begin{exercise}
  设$f_n(x)=\underbrace{f\{f[\cdots f(x)]\}}_{n\text{次}}$,若$f(x)=\dfrac{x}{\sqrt{1+x^2}}$,猜想$f_n(x)$.
\end{exercise}

\begin{solution}
  $f_{n}(x)=\dfrac{x}{\sqrt{1+nx^{2}}}$
\end{solution}

\begin{exercise}
  设二次函数$f(x)=ax^{2}+bx+c(a>0)$,方程$f(x)-x=0$的两根$x_1,x_2$满足$0<x_1<x_2<\dfrac1a$.
\end{exercise}

\begin{enumerate}
  \item 当$x\in(0,x_1)$时,求证:$x<f(x)<x_1$;
  \item 设函数$f(x)$的图像关于$x=x_0$对称,求证:$x_0<\dfrac{x_1}{2}$.
\end{enumerate}

\begin{solution}
  待补充.
\end{solution}

\begin{exercise}
  在定义域$[a,b]$上递减的函数$f(x)$的最大值是多少?
\end{exercise}

\begin{solution}
  $f(a)$.
\end{solution}

\begin{exercise}
  若$f(x)$在区间$[a,b]$上递减而在区间$[b,c]$上递增,那么$f(x)$的最小值是多少?
\end{exercise}

\begin{solution}
  $f(b)$.
\end{solution}

\begin{exercise}
  *设函数$f(x)$的定义域为$D$,证明:$f(x)$在$A\subseteq D$上有界的充要条件是它在$A$上既有上界又有下界.
\end{exercise}

\begin{spacing}{1.8}
  \begin{exercise}
    证明$f(x)=\dfrac x{x^2+1}$在$(-1,1)$上是增函数.
  \end{exercise}
  
  \begin{exercise}
    证明$f(x)=x^3$在$\mathbb{R}$上是增函数.
  \end{exercise}
  
  \begin{exercise}
    求下列函数的单调区间,并使用Geogebra验证(在这时Geogebra比Mathematica更直观).
  \end{exercise}
  
  \begin{enumerate}
    \item $f(x)=\dfrac{1}{\sqrt{x^{2}+x+1}}$;
    \item $f(x)=\dfrac{1}{x-2}$;
    \item $f(x)=\dfrac1{\sqrt{x}}$;
    \item $f(x)=x^{2}-2|x|-1$.
  \end{enumerate}
\end{spacing}

\begin{exercise}
  已知函数$f(x)$为奇函数,当$x>0$时,$f(x)=x^2+\frac1x$,求$f(-1)$.
\end{exercise}

\begin{exercise}
  因为$f(x)$是奇函数,所以$f(-1)=-f(1)=-(1^2+\frac11)=-2$.
\end{exercise}

\begin{exercise}
  *已知函数$f(x),g(x)$在公共的区间$A$上都是增(减)函数,求证:$f(x)+g(x)$在$A$上也是增(减)函数.
\end{exercise}

\begin{exercise}
  *证明下列二者之一:
\end{exercise}

\begin{enumerate}
  \item 若两个正值函数$f(x)$和$g(x)$在公共区间$A$内都是增(减)函数,则函数$f(x)\cdot g(x)$在区间$A$内也是增(减)函数;
  \item 若两个负值函数$f(x)$和$g(x)$在公共区间$A$内都是增(减)函数,则函数$f(x)\cdot g(x)$在区间 $A$ 内是减(增)函数.
\end{enumerate}

\begin{exercise}
  *证明:偶函数的图像关于$y$轴对称,奇函数的图像关于原点对称.
\end{exercise}

\begin{spacing}{1.8}
  \begin{exercise}
    证明:存在唯一的函数$f(x)$,它既是奇函数又是偶函数.
  \end{exercise}

  \begin{exercise}
    判断下列函数是否具有奇偶性:
  \end{exercise}

  \begin{enumerate}
    \item $f(x)=2x^2+|x|$;
    \item $h(x)=x^3(x\in[-1,3])$;
    \item $f(x)=x+\dfrac{x^3}7-\dfrac{x^7}{10}$;
    \item $f(x)=(x+1)(x-1)$;
    \item $f(x)=\dfrac{1}{x^{2}-1}$;
    \item $f(x)=\begin{cases}x-1, x\geqslant0,\\-x-1, x<0\end{cases}$.
  \end{enumerate}
\end{spacing}

\begin{exercise}
  *设$f(x),g(x)$分别为定义域是$D_1,D_2$的奇函数,证明:在 $D_1\cap D_2$上$,f(x)+g(x)$是奇函数$,f(x)\cdot g(x)$是偶函数.
\end{exercise}

\begin{remark}
  全部情况:奇$\pm$奇=奇,奇$\times$奇=偶,偶士偶=偶,偶×偶=偶,奇×偶=奇.
\end{remark}

\begin{exercise}
  已知$f(x),g(x)$分别是定义在$\mathbb{R}$上的偶函数和奇函数,且$f(x)-g(x)=x^3+x^2+1$,求$f(1)+g(1)$.
\end{exercise}

\begin{solution}
  因为$f(x),g(x)$分别是定义在$\mathbb{R}$上的偶函数和奇函数,所以$f(x)+g(x)=f(-x)-g(-x)=-x^3+x^2+1$.所以$f(1)+g(1)=1$.
\end{solution}

\begin{exercise}
  *证明:若$f(x)$是严格单调函数,则当$f(x)$是奇(偶)函数时,它在正负对称的区间上(如$(0,1)$和$(-1,0)$)的单调性是相同(反)的.
\end{exercise}

\begin{exercise}
  *对于复合函数$\phi(x)=f[g(x)]$,证明下列三者之一:
\end{exercise}

\begin{enumerate}
  \item 若$g(x)$为偶函数,则$\phi(x)$为偶函数;
  \item 若$f(x),g(x)$均为奇函数,则$\phi(x)$为奇函数;
  \item 若$g(x)$为奇函数,$f(x)$为偶函数,则$\phi(x)$为偶函数.
\end{enumerate}

\begin{exercise}
  证明:任一定义域关于原点对称的函数都可以写成一个奇函数和一个偶函数的和.
\end{exercise}

\begin{exercise}\label{2017RJB_bx1_P111.B7}
  已知函数$f(x)=ax^2-2ax-3(a>0)$,运用函数性质来比较$f(-2)$与$f(4),f(-3)$与$f(3)$的大小.
\end{exercise}

\begin{exercise}\label{2017RJB_bx1_P111.B8}
  已知函数$f(x)=(x-1)^2+ax+2$是偶函数,求实数$a$的值.
\end{exercise}

\begin{exercise}\label{2017RJB_bx1_P111.C2}
  若函数$f(x)$是定义在$\mathbb{R}$上的偶函数,且$f(x)$在$(-\infty,0]$上严格单调递减,若$f(2)=0$,求不等式$f(x)<0$的解集.
\end{exercise}

\begin{exercise}\label{ASJC_G1_P30.11}
  设$x,y\in\mathbb{R}$,且$\begin{cases}(x-1)^3+2008(x-1)=-1,\\(y-1)^3+2008(y-1)=1.\end{cases}$,求$x+y$的值.
\end{exercise}

\begin{exercise}\label{HS2FZ_lkb1_P57.3}
  设$f(x)$是定义在$\mathbb{R}$上的函数,且满足$f(10+x)=f(10-x),f(20+x)=-f(20-x)$,证明:$f(x)$是奇函数,且$f(x)$是周期函数.
\end{exercise}

\begin{exercise}\label{HS2FZ_lkb1_P55.2}
  设函数$f(x)$的定义域为$\mathbb{R}$,实数$a>0$,且对任意$x\in\mathbb{R}$,有$f(x+a)=\dfrac{1}{2}+\sqrt{f(x)-[f(x)]^{2}}$,证明:$f(x)$是周期函数.
\end{exercise}

\begin{exercise}
  对指数为零和负整数的情况证明指数幂运算性质.
\end{exercise}

\begin{spacing}{1.8}
  \begin{exercise}\label{ZXSXSYJC_reformatted_3_P9.8b}
  计算$\left[\left(\dfrac{5a^{-2}c^3}{3x^{-3}y^4}\right)^{-2}\right]^4$.
  \end{exercise}

  \begin{exercise}\label{ZXSXSYJC_reformatted_3_P9.9b}
  计算$\dfrac{a^{-3}+b^{-3}}{a^{-1}+b^{-1}}+\dfrac{a^{-3}-b^{-3}}{a^{-1}-b^{-1}}$.
  \end{exercise}
\end{spacing}

\begin{exercise}
  计算下列式子:
\end{exercise}

\begin{spacing}{1.8}
  \begin{enumerate}
    \item $16^{-\frac{3}{2}}$;$\left(-0.25\right)^{-1}$;$4^{\frac32}$;
    \item $\sqrt[5]{(3-\sqrt{2})^5}$;
    \item $2^{-1+\sqrt{3}}\times16^{-\dfrac{\sqrt{3}}4}$;
    \item $\sqrt{5-2\sqrt6}+\sqrt{5+2\sqrt6}$;
    \item $(0.0625)^{\frac{1}{4}}-\left[-2\times\left(\dfrac{7}{3}\right)^{0}\right]^{2}\times\left[\left(-2\right)^{3}\right]^{\frac{4}{3}}+10\left(2-\sqrt{3}\right)^{-1}-\left(\frac{1}{300}\right)^{-0.5}$;
    \item $\dfrac{\left(2^{n+1}\right)^2\times\left(\dfrac{1}{2}\right)^{2n+1}}{4^n\times8^{-2}}$;
    \item $4a^{\frac{2}{3}}b^{-\frac{1}{3}}\div\left(-\dfrac{2}{3}a^{-\frac{1}{3}}b^{-\frac{1}{3}}\right)$;
    \item $\sqrt[3]{a^{\frac{7}{2}}\cdot\sqrt{a^{-3}}}\div\sqrt[3]{a^{-8}\cdot\sqrt[3]{a^{15}}}\div\sqrt[3]{\sqrt{a^{-3}}\cdot\sqrt{a^{-1}}}$;
    \item $a\sqrt{a\sqrt{a\sqrt{a}}}$;
    \item $\sqrt{\dfrac{y^{2}}{x}\cdot\sqrt{\dfrac{x^{3}}{y}\cdot\sqrt[3]{\dfrac{y^{6}}{x^{3}}}}}$;
    \item $\dfrac{a+b}{\sqrt{a}+\sqrt{b}}+\dfrac{2ab}{a\sqrt{b}+b\sqrt{a}}$.
  \end{enumerate}
\end{spacing}

\begin{solution}
  $\dfrac{1}{64};-4;8;3-\sqrt{2};\dfrac12;2\sqrt{3};-\dfrac{87}{2};\left(\dfrac12\right)^{2n-7};-6a;a^{\frac{15}{8}};a^{\frac16};\sqrt{a}+\sqrt{b}.$
\end{solution}

\begin{exercise}\label{2017RJA_bx1_P109.2.2}
  设$a>0$,$m,n$是正整数,且$n>1$,则$a^{\frac{m}{n}}=\sqrt[n]{a^m} , a^0=1 , a^{-\frac{m}{n}}=\dfrac{1}{\sqrt[n]{a^m}}$中运算正确的有\underline{\hbox to 10mm{}}项.
\end{exercise}

\begin{solution}
  3.
\end{solution}

\begin{exercise}\label{2017RJA_bx1_P110.7.1}
  已知$10^m=2,10^n=3$,求$10^{\frac{3m-2n}{2}}$的值.
\end{exercise}

\begin{solution}
  $\dfrac{2\sqrt{2}}{3}$.
\end{solution}

\begin{exercise}
  已知$a^\frac12+a^{-\frac12}=3$,则$a+a^{-1}=$\underline{\hbox to 10mm{}},$a^{2}+a^{-2}=$\underline{\hbox to 10mm{}}.
\end{exercise}

\begin{solution}
  7;47.
\end{solution}

\end{document}
