\documentclass[lang=cn,newtx,10pt,scheme=chinese]{elegantbook}

\title{数学之桥:从初等数学到微积分}
\subtitle{习题答案}

\author{Siliconet}
\date{始于2023年}
\version{1.0}
\bioinfo{模板}{ElegantBook}

\extrainfo{你站在桥上看风景,看风景人在楼上看你。明月装饰了你的窗子,你装饰了别人的梦。}

\setcounter{tocdepth}{3}

\cover{cover.jpg}

% 本文档命令
\usepackage{array}
\newcommand{\ccr}[1]{\makecell{{\color{#1}\rule{1cm}{1cm}}}}

% 修改标题页的橙色带
\definecolor{customcolor}{RGB}{255,255,255}
\colorlet{coverlinecolor}{customcolor}
\usepackage{cprotect}

\addbibresource[location=local]{reference.bib} % 参考文献,不要删除

\begin{document}

\maketitle
\frontmatter

\tableofcontents

\mainmatter

\chapter{集合论初步}
\begin{exercise}
  $\in\enspace\notin\enspace\notin\enspace\in\enspace\notin$.
\end{exercise}
\begin{exercise}
  无限集;有限集;无限集。
\end{exercise}
\begin{exercise}
  $\{2,-2\}$;$\{-2,-1,0,1,2\}$;$\{1\}$.
\end{exercise}
\begin{exercise}
  (1)$\{(x,y)\mid x>0,y>0\}$;(2)$\{\text{矩形}\}$;(3)$\{x\mid x=5^n,0\leq n\leq 4,n\in\mathbb{N}\}$.
\end{exercise}
\begin{exercise}
  $\{(2,1,3)\}$.
\end{exercise}
\begin{exercise}
  $-1$或$-8$。
\end{exercise}
\begin{exercise}
  (1)线段$AB$的垂直平分线;(2)以$O$为圆心,$3cm$为半径的圆;(3)以$O$为圆心,内半径为2,外半径为3的圆环。
\end{exercise}
\begin{exercise}
  (1)$\{y|y\geqslant0\}$或$[0,+\infty)$;(2)$\{y|y\neq0\}$;(3)$\{y\text{轴上的点}\}$
\end{exercise}
\begin{exercise}
  $m\geqslant2\text{或}m=1$.
\end{exercise}
\begin{exercise}
  983.
\end{exercise}
\begin{exercise}
  略。
\end{exercise}
\begin{exercise}
  略。
\end{exercise}
\begin{exercise}
  不相等.$\varnothing\subseteq\{0\}.$
\end{exercise}
\begin{exercise}
  略。
\end{exercise}
\begin{exercise}
  $\overline{A}=\{4,5\};\overline{A}\cap B=\{3\};A\cup\overline{B}=\{1,2,3,4,6\}.$
\end{exercise}
\begin{exercise}
  $\{4\};\{1,2,4,5,6,7,8,9,10\};\{3\};\{1,2,3,5,6,7,8,9,10\};\varnothing ;\{1,2,3,4,5,6,7,8,9,10\}.$
\end{exercise}
\begin{exercise}
  $[4,+\infty).$
\end{exercise}
\begin{exercise}
  略。
\end{exercise}
\begin{exercise}
  (1)(2)略。
  (3)60.
\end{exercise}
\begin{exercise}
  $\{-3,-1,1,5\}$.
\end{exercise}
\begin{exercise}
  (1)设$s=m_1^2+n_1^2,t=m_2^2+n_2^2$,其中$m_1,m_2,n_1,n_2$均为整数。由此可得
  $$\begin{aligned}
    st& =(m_1^2+n_1^2)(m_2^2+n_2^2)  \\
    &=m_1^2m_2^2+n_1^2n_2^2+m_1^2n_2^2+m_2^2n_1^2 \\
    &=(m_1m_2+n_1n_2)^2+(m_1n_2-m_2n_1)^2
    \end{aligned}$$
  因此$st\in A$,得证。
  
  (2)由(1)知$st \in A$,设$st=m^2+n^2$,其中$m,n$都是整数。由于$t\neq 0$,可得$$\frac st=\frac{st}{t^2}=\frac{m^2+n^2}{t^2}=\left(\frac mt\right)^2+\left(\frac nt\right)^2$$
  得证。
\end{exercise}
\begin{exercise}\label{exer:202405021028}
  试着举出一个真命题,它的逆命题不是真命题。
\end{exercise}
\begin{solution}
  略。
\end{solution}
\begin{exercise}
  在命题“三角形的内角和等于$180^{\circ}$”中,条件和结论分别是什么?
\end{exercise}
\begin{solution}
  条件:一个平面图形是三角形;结论:这个平面图形的内角和为$180^{\circ}$.
\end{solution}
\begin{example}\label{exp:202406231441}
  幼儿园的小朋友都知道,$4>2$,所以,$x>4\Rightarrow x>2$.
\end{example}
\begin{problem}\label{202406262000}
  参考例\ref{exp:202406231441},思考怎样从集合的观点理解推出关系。
\end{problem}
\begin{solution}
  略。
\end{solution}
\begin{exercise}\label{202406262010}
  判断下列各题中,$p$是否是$q$的充分条件,$q$是否是$p$的必要条件:

  (1)$p:x\in\mathbb{Z},q:x\in\mathbb{R}$;

  (2)$p:a=b,q:ac=bc$;

  (3)$p:$两个三角形全等,$q:$两个三角形面积相等.
\end{exercise}
\begin{solution}
  (1)是;
  (2)是;
  (3)是。
\end{solution}
\begin{exercise}\label{202406262101}
  判断下列“若p,则q”形式的命题当中,哪些命题中的p是q的充分条件,或者说q是p的必要条件?

  (1)若四边形的两组对角分别相等,则这个四边形是平行四边形;

  (2)若四边形为菱形,则这个四边形的对角线互相垂直;

  (3)若$x,y$为无理数,则$xy$为无理数;

  (4)若直线$l$与$\odot O$有且仅有一个交点,则$l$为$\odot O$的一条切线;

  (5)若$x,y$都能被5整除,则$x+y$也能被5整除;

  (6)如果$\dfrac{a}{b}=\dfrac{c}{d}$,则$\dfrac{a+b}{b}=\dfrac{c+d}{d}$.
\end{exercise}
\begin{solution}
  (1)(2)(4)(5)(6)
\end{solution}
\begin{exercise}\label{RJB_P26}
  判断下列命题的真假:

  (1)存在两个无理数,它们的乘积是有理数;

  (2)没有一个无理数不是实数;

  (3)集合$A$是集合$A\cup B$的子集;

  (4)集合$A\cap B$是集合$A$的子集;

  (5)如果$\dfrac ab=\dfrac cd$,则$\dfrac{a+b}b=\dfrac{c+d}{d}$;

  (6)一元三次方程都有三个不同的实数根;
  
  (7)如果$\dfrac ab=\dfrac cd\neq 1$,则$\dfrac a{b-a}=\dfrac c{d-c}$.
\end{exercise}
\begin{solution}
  真;真;真;真;真;假;真。
\end{solution}
\begin{exercise}\label{202407081520}
  证明:$\sqrt{3}$是无理数。
\end{exercise}
\begin{solution}
  略。
\end{solution}
\begin{exercise}\label{BJ4Z_Algebra1_P28.2}
  用“充要”“充分不必要”“必要不充分”“既不充分也不必要”填空(本题中默认$a\in\mathbb{R},b\in\mathbb{R}$):

  (1)$a^2=b^2$是$a=b$的\underline{\hbox to 15mm{}}条件;

  (2)$a^3=b^3$是$a=b$的\underline{\hbox to 15mm{}}条件;

  (3)使$ab=0$的充分条件是\underline{\hbox to 15mm{}};

  (4)使$ab=0$的必要条件是\underline{\hbox to 15mm{}};

  (5)使$ab\neq 0$的充要条件是\underline{\hbox to 15mm{}};

  (6)“$a>2$”是“$a\geqslant2$”的\underline{\hbox to 15mm{}}条件;

  (7)“$a、b$不全是0”是“$a、b$全不是0”的\underline{\hbox to 15mm{}}条件;

  (8)“$\vert a-2\vert\neq2-a$”是“$a\geqslant2$”的\underline{\hbox to 15mm{}}条件。
\end{exercise}
\begin{solution}
  必要不充分;充要;$a=0$(仅举一例);$a=0$或$b=0$;$a\neq 0$且$b\neq 0$;充分不必要;既不充分也不必要;充要。
\end{solution}

\begin{exercise}\label{2017_XJ_bx1_P23.8}
  已知$p$是$r$的充分条件,$r$是$q$的必要条件,同时也是$s$的充分条件,$q$是$s$的必要条件,那么:
\end{exercise}

\begin{enumerate}
  \item $s$是$p$的什么条件?
  \item $p$是$q$的什么条件?
  \item 在$p,q,r,s$中,哪几对互为充要条件?
\end{enumerate}

\begin{solution}
  必要条件;充分条件;$r$与$q$,$r$与$s$,$q$与$s$.
\end{solution}

\begin{exercise}\label{2017_RJB_bx1_P36.B5}
  已知$A=(-\infty,a]$,$B=(-\infty,3)$,且$x\in A$是$x\in B$的充分不必要条件,求$a$的取值范围。
\end{exercise}

\begin{solution}
  $(-\infty,3)$.
\end{solution}

\begin{exercise}\label{zhw2000_g1_P51.78}
  求证:关于$x$的方程$ax^2+bx+1=0$有一根为1的充要条件是$a+b+c=0$.
\end{exercise}

\begin{solution}
  必要性证明略,下面证明充分性。

  若$a+b+c=0$,则原方程可化为$ax^2+bx-(a+b)=0$,即$(x-1)(ax+a+b)=0$,由此可知原方程有一根为1.充分性证毕。
\end{solution}

\hspace*{\fill}
\begin{itemize}
  \item 两个命题互为逆否命题,它们的真假性相同;
  \item 两个命题为互逆命题或互否命题,它们的真假性无关。
\end{itemize}
\begin{problem}
  举出这样的例子。
\end{problem}
\begin{solution}
  略。
\end{solution}

\begin{exercise}
  用数学符号写出下列命题并判断真假:
\end{exercise}

\begin{enumerate}
  \item 所有实数的平方都是正数;
  \item 存在两个无理数,它们的乘积是有理数;
  \item 三个连续整数的乘积是6的倍数;
  \item 大于3的自然数是不等式$x^2>10$的解。
\end{enumerate}

\begin{solution}
  $\forall x\in\mathbb{R},x^2>0$,假;$\exists a,b\in\complement_{\mathbb{R}}\mathbb{Q},ab\in\mathbb{Q}$,真;$\forall a\in\mathbb{N},a(a-1)(a+1)=6k\text{且}k\in\mathbb{Z}$,假;$\forall x>3\text{且}x\in\mathbb{N},x^2>10$,真。
\end{solution}

\begin{exercise}
  判断下列命题的真假,并写出这些命题的否定(请注意这里有些命题\textcolor{blue}{省略了全称量词}):
\end{exercise}

\begin{enumerate}
  \item 至少有一个整数$n$,$n^2+1$是4的倍数;
  \item 每个二次函数的图像都是轴对称图形;
  \item 存在一个四边形,它的四个顶点不共圆;
  \item 若$x>1$,则$2x+1>5$;
  \item 若四边形为等腰梯形,则这个四边形的对角线相等;
  \item $\forall x{\in}\mathbb{R}$,$\dfrac1{x^2+1}<1$;
  \item $\forall a,b{\in}\mathbb{R}$,$a^3+b^3=(a+b)(a^2-ab+b^2)$.
\end{enumerate}

\begin{solution}
  
\end{solution}

\begin{enumerate}
  \item 假;对所有的整数$n$,$n^2+1$不是4的倍数;
  \item 真;至少有一个二次函数的图像不是轴对称图形;
  \item 真;任意一个四边形的四个顶点共圆;
  \item 假;存在$x>1$,使得$2x+1\leqslant 5$;
  \item 真;存在一个为等腰梯形的四边形,它的对角线不相等;
  \item 假;$\exists x\in\mathbb{R},\dfrac1{x^2+1}\geqslant 1$;
  \item 真;$\exists a,b{\in}\mathbb{R}$,$a^3+b^3\neq(a+b)(a^2-ab+b^2)$.
\end{enumerate}

\begin{exercise}
  判断下列命题的真假:
\end{exercise}

\begin{enumerate}
  \item $\forall x\in(-1, 2), x\in[-1, 2)$;
  \item $\forall x\in\mathbb{Q}, |x|+x\geqslant0$;
  \item $\exists x,y\in\mathbb{Z},3x-2y=10$;
  \item 集合$A$是集合$A\cup B$的子集;
  \item 集合$A\cap B$是集合$A$的子集;
  \item 存在有序整数组$(x,y)$满足$xy=x+y$.
\end{enumerate}

\begin{solution}
  真;真;真;真;真;真。
\end{solution}

\begin{exercise}
  用联结词“且”和“或”分别联结下面各组命题组成新命题,并判断它们的真假:
\end{exercise}

\begin{enumerate}
  \item $p$:$17<20$;$q$:$17=20$.
  \item $p$:平行四边形对角线互相平分;$q$:平行四边形的对角线相等。
  \item $p$:4是素数;$q$:4不是奇数;
  \item $p$:能被5整除的整数的个位数一定为5;$q$:能被5整除的整数的个位数一定为0。
\end{enumerate}

\begin{solution}
  略;略;略;能被5整除的整数的个位数一定为5或一定为0,假。
\end{solution}

\begin{exercise}
  写出下列命题的非:
\end{exercise}

\begin{enumerate}
  \item $\forall x\in\mathbb{R},3x=2x+x$;
  \item $\exists x\in\mathbb{N},x^2=x+2$;
  \item 至少有一个锐角$\alpha$,使$\sin\alpha=0$;
  \item $a,b$都是有理数。
\end{enumerate}\

\begin{solution}
  略;略;对于任意的锐角$\alpha$,$\sin\alpha\neq 0$;$a,b$不全是有理数。
\end{solution}

\begin{exercise}
  写出下列命题的逆命题、否命题和逆否命题并判断真假:
\end{exercise}

\begin{enumerate}
  \item 线段的垂直平分线上的点到这条线段两个端点的距离相等;
  \item 矩形的对角线相等;
  \item 如果$x=y=0$,那么$(x-y)(x+y)=0$;
  \item 对顶角相等。
\end{enumerate}

\begin{solution}
  略。
\end{solution}

\begin{exercise}\label{2003RJA_xx2-1_P7.exp4}
  证明:若$x^2+y^2=0$,则$x=y=0$.
\end{exercise}

\begin{solution}
  取逆否命题即可。
\end{solution}

\end{document}
